\section{Literature Review}
Firstly, causal ML methods are powerful tools in using data to recover
complex interactions among variables and flexibly estimate the relationship between the outcome, the treatment indicator and covariates.

Secondly, causal ML methods allow for the inclusion of a large number of
covariates, even when the sample size is relatively small, by assuming that the model is sparse, (i.e., only a small number of covariates are relevant), and using regularized regressions.

Thirdly, the use of causal ML methods allows to implement systematic
model selection.

Finally, causal machine learning methods prove to be very useful when
one is interested in estimating heterogeneous treatment effects.



\subsection*{Specific Machine Learning}
move the target from the estimation of outcomes to the estimation of IATEs
Causal Forests are expected to perform well in low-dimensional settings, where the regression functions can be well approximated by trees (Wager and Athey, 2018). The generic approaches that
combine standard regression Random Forests could theoretically work also in higher dimensions (Wager and Walther, 2015)
\subsection*{Generic Machine Learning}
consider only estimators that require at most one additional estimation step on top of the estimation of the nuisance parameters, owing to restrictions in computation power.

three conditional outcome difference methods proposed by